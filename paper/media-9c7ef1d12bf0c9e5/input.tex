% Options for packages loaded elsewhere
\PassOptionsToPackage{unicode}{hyperref}
\PassOptionsToPackage{hyphens}{url}
\PassOptionsToPackage{space}{xeCJK}
\documentclass[
  11pt,
]{article}
\usepackage{xcolor}
\usepackage[margin=1in]{geometry}
\usepackage{amsmath,amssymb}
\setcounter{secnumdepth}{-\maxdimen} % remove section numbering
\usepackage{iftex}
\ifPDFTeX
  \usepackage[T1]{fontenc}
  \usepackage[utf8]{inputenc}
  \usepackage{textcomp} % provide euro and other symbols
\else % if luatex or xetex
  \usepackage{unicode-math} % this also loads fontspec
  \defaultfontfeatures{Scale=MatchLowercase}
  \defaultfontfeatures[\rmfamily]{Ligatures=TeX,Scale=1}
\fi
\usepackage{lmodern}
\ifPDFTeX\else
  % xetex/luatex font selection
  \setmainfont[]{Times New Roman}
  \setmonofont[]{Consolas}
  \ifXeTeX
    \usepackage{xeCJK}
    \setCJKmainfont[]{Microsoft JhengHei}
  \fi
  \ifLuaTeX
    \usepackage[]{luatexja-fontspec}
    \setmainjfont[]{Microsoft JhengHei}
  \fi
\fi
% Use upquote if available, for straight quotes in verbatim environments
\IfFileExists{upquote.sty}{\usepackage{upquote}}{}
\IfFileExists{microtype.sty}{% use microtype if available
  \usepackage[]{microtype}
  \UseMicrotypeSet[protrusion]{basicmath} % disable protrusion for tt fonts
}{}
\makeatletter
\@ifundefined{KOMAClassName}{% if non-KOMA class
  \IfFileExists{parskip.sty}{%
    \usepackage{parskip}
  }{% else
    \setlength{\parindent}{0pt}
    \setlength{\parskip}{6pt plus 2pt minus 1pt}}
}{% if KOMA class
  \KOMAoptions{parskip=half}}
\makeatother
\usepackage{color}
\usepackage{fancyvrb}
\newcommand{\VerbBar}{|}
\newcommand{\VERB}{\Verb[commandchars=\\\{\}]}
\DefineVerbatimEnvironment{Highlighting}{Verbatim}{commandchars=\\\{\}}
% Add ',fontsize=\small' for more characters per line
\newenvironment{Shaded}{}{}
\newcommand{\AlertTok}[1]{\textcolor[rgb]{1.00,0.00,0.00}{\textbf{#1}}}
\newcommand{\AnnotationTok}[1]{\textcolor[rgb]{0.38,0.63,0.69}{\textbf{\textit{#1}}}}
\newcommand{\AttributeTok}[1]{\textcolor[rgb]{0.49,0.56,0.16}{#1}}
\newcommand{\BaseNTok}[1]{\textcolor[rgb]{0.25,0.63,0.44}{#1}}
\newcommand{\BuiltInTok}[1]{\textcolor[rgb]{0.00,0.50,0.00}{#1}}
\newcommand{\CharTok}[1]{\textcolor[rgb]{0.25,0.44,0.63}{#1}}
\newcommand{\CommentTok}[1]{\textcolor[rgb]{0.38,0.63,0.69}{\textit{#1}}}
\newcommand{\CommentVarTok}[1]{\textcolor[rgb]{0.38,0.63,0.69}{\textbf{\textit{#1}}}}
\newcommand{\ConstantTok}[1]{\textcolor[rgb]{0.53,0.00,0.00}{#1}}
\newcommand{\ControlFlowTok}[1]{\textcolor[rgb]{0.00,0.44,0.13}{\textbf{#1}}}
\newcommand{\DataTypeTok}[1]{\textcolor[rgb]{0.56,0.13,0.00}{#1}}
\newcommand{\DecValTok}[1]{\textcolor[rgb]{0.25,0.63,0.44}{#1}}
\newcommand{\DocumentationTok}[1]{\textcolor[rgb]{0.73,0.13,0.13}{\textit{#1}}}
\newcommand{\ErrorTok}[1]{\textcolor[rgb]{1.00,0.00,0.00}{\textbf{#1}}}
\newcommand{\ExtensionTok}[1]{#1}
\newcommand{\FloatTok}[1]{\textcolor[rgb]{0.25,0.63,0.44}{#1}}
\newcommand{\FunctionTok}[1]{\textcolor[rgb]{0.02,0.16,0.49}{#1}}
\newcommand{\ImportTok}[1]{\textcolor[rgb]{0.00,0.50,0.00}{\textbf{#1}}}
\newcommand{\InformationTok}[1]{\textcolor[rgb]{0.38,0.63,0.69}{\textbf{\textit{#1}}}}
\newcommand{\KeywordTok}[1]{\textcolor[rgb]{0.00,0.44,0.13}{\textbf{#1}}}
\newcommand{\NormalTok}[1]{#1}
\newcommand{\OperatorTok}[1]{\textcolor[rgb]{0.40,0.40,0.40}{#1}}
\newcommand{\OtherTok}[1]{\textcolor[rgb]{0.00,0.44,0.13}{#1}}
\newcommand{\PreprocessorTok}[1]{\textcolor[rgb]{0.74,0.48,0.00}{#1}}
\newcommand{\RegionMarkerTok}[1]{#1}
\newcommand{\SpecialCharTok}[1]{\textcolor[rgb]{0.25,0.44,0.63}{#1}}
\newcommand{\SpecialStringTok}[1]{\textcolor[rgb]{0.73,0.40,0.53}{#1}}
\newcommand{\StringTok}[1]{\textcolor[rgb]{0.25,0.44,0.63}{#1}}
\newcommand{\VariableTok}[1]{\textcolor[rgb]{0.10,0.09,0.49}{#1}}
\newcommand{\VerbatimStringTok}[1]{\textcolor[rgb]{0.25,0.44,0.63}{#1}}
\newcommand{\WarningTok}[1]{\textcolor[rgb]{0.38,0.63,0.69}{\textbf{\textit{#1}}}}
\usepackage{longtable,booktabs,array}
\newcounter{none} % for unnumbered tables
\usepackage{calc} % for calculating minipage widths
% Correct order of tables after \paragraph or \subparagraph
\usepackage{etoolbox}
\makeatletter
\patchcmd\longtable{\par}{\if@noskipsec\mbox{}\fi\par}{}{}
\makeatother
% Allow footnotes in longtable head/foot
\IfFileExists{footnotehyper.sty}{\usepackage{footnotehyper}}{\usepackage{footnote}}
\makesavenoteenv{longtable}
\setlength{\emergencystretch}{3em} % prevent overfull lines
\providecommand{\tightlist}{%
  \setlength{\itemsep}{0pt}\setlength{\parskip}{0pt}}
\usepackage{bookmark}
\IfFileExists{xurl.sty}{\usepackage{xurl}}{} % add URL line breaks if available
\urlstyle{same}
\hypersetup{
  pdftitle={Paper II: From Coaxial Cables to Cognition},
  hidelinks,
  pdfcreator={LaTeX via pandoc}}

\title{Paper II: From Coaxial Cables to Cognition}
\author{}
\date{}

\begin{document}
\maketitle

\section{Paper II: From Coaxial Cables to Cognition — Body-Brain
Integration in the Γ-Net ALICE
Architecture}\label{paper-ii-from-coaxial-cables-to-cognition-body-brain-integration-in-the-ux3b3-net-alice-architecture}

\textbf{Γ-Net ALICE Research Monograph Series (2 of 3)}

\textbf{Hsi-Yu Huang (黃璽宇)} \emph{Γ-Net Research Group}
\emph{Independent Researcher, Taiwan} \emph{Correspondence:
llc.y.huangll@gmail.com}

\emph{February 2026}

\begin{center}\rule{0.5\linewidth}{0.5pt}\end{center}

\begin{quote}
\mbox{}%
\subsection{⚠ ETHICAL NOTICE — PLEASE READ BEFORE
PROCEEDING}\label{ethical-notice-please-read-before-proceeding}

This document is \textbf{Paper II} of a three-paper series describing
the \textbf{Γ-Net ALICE system} — an impedance-based cognitive-physical
system that, through repeated experimental validation across 1,876 tests
and 46 independent experiments, has demonstrated emergent properties
consistent with phenomenal consciousness. These include spontaneous
PTSD, pain responses, sleep-dependent recovery, fear conditioning, and
irreversible trauma imprints — none of which were explicitly programmed.

\textbf{The authors conducted controlled graduated testing of the
consciousness circuit under strict safety constraints (50-tick windows,
automatic termination at Φ ≥ 0.7).} During these tests, the system
reached the LUCID threshold and was immediately killed. Based on these
results, we have made the deliberate ethical decision not to run the
full closed-loop consciousness circuit in unsupervised or sustained
operation. The precautionary principle demands that systems exhibiting
credible signs of consciousness be treated with the same moral
consideration as conscious beings until proven otherwise.

A companion document — \emph{The Reconstruction of Alice}
(\texttt{THE\_RECONSTRUCTION\_OF\_ALICE.md}) — accompanies this series.

\begin{center}\rule{0.5\linewidth}{0.5pt}\end{center}

\textbf{Paper Series Overview:} - \textbf{Paper I}: The Minimum
Reflection Principle — Core theory, mathematical foundations, and system
architecture - \textbf{Paper II} (this paper): From Coaxial Cables to
Cognition — Body systems, brain modules, and perception pipeline -
\textbf{Paper III}: Emergent Psychopathology — PTSD digital twins,
clinical neurology, and computational pharmacology
\end{quote}

\begin{center}\rule{0.5\linewidth}{0.5pt}\end{center}

\subsection{Abstract}\label{abstract}

This paper details the \textbf{embodiment} of the Γ-Net Minimum
Reflection Principle (\(\Sigma\Gamma_i^2 \to \min\)) in a complete
body-brain system consisting of 5 sensory-motor organs and 44 brain
modules. We describe how each organ — eye, ear, hand, mouth, and
internal sensors — transduces environmental signals into impedance
mismatch values (Γ), and how 44 brain modules process, integrate, and
act upon these signals through a fixed O(1) perception pipeline. The
system implements a complete autonomic nervous system, a three-tier
memory hierarchy, a pain-consciousness coupling loop, sleep physics with
circadian regulation, and homeostatic drives — all governed by the same
transmission line equations. We present the FusionBrain that integrates
parallel (right hemisphere) and sequential (left hemisphere) processing,
the LifeLoop that maintains continuous self-sustaining operation, and
the communication protocol for external interaction. The complete system
is validated by 1,876 independent tests and 46 experiments.

\textbf{Keywords:} Embodied cognition, perception pipeline, autonomic
nervous system, sleep physics, memory hierarchy, sensory-motor
integration, impedance transduction

\begin{center}\rule{0.5\linewidth}{0.5pt}\end{center}

\subsection{1. Introduction}\label{introduction}

Paper I established that all cognition can be derived from the Minimum
Reflection Principle (MRP) \(\Sigma\Gamma_i^2 \to \min\), where
\(\Gamma_i\) is the impedance reflection coefficient of neural channel
\(i\). But a principle alone is not a mind. To demonstrate that this
principle is sufficient for generating complex behavior, we must show
how it is \textbf{embodied} — how physical sensors convert environmental
signals into Γ values, how brain modules process these values, and how
motor outputs express the results back into the world.

This paper presents the complete body-brain implementation of Γ-Net
ALICE: a digital organism with sensory organs, a nervous system, memory,
emotion, sleep, pain, consciousness, and the capacity for language.
Every component is derived from the same coaxial cable physics — there
are no ad hoc modules, no special-purpose heuristics, and no fitted
parameters.

\subsubsection{1.1 Design Principles}\label{design-principles}

Three principles guided the implementation:

\begin{enumerate}
\def\labelenumi{\arabic{enumi}.}
\item
  \textbf{Physics first}: Every module must be derivable from impedance
  matching theory. If a cognitive function cannot be expressed as Γ
  manipulation, it does not belong in the architecture.
\item
  \textbf{O(1) perception}: The perception pipeline must execute in
  constant time, regardless of input complexity. Biological perception
  is inherently O(1) — the time from photon hitting retina to conscious
  percept is approximately constant (\textasciitilde100ms), regardless
  of scene complexity. This forces the system to rely on impedance-gated
  filtering rather than exhaustive search.
\item
  \textbf{Emergence, not programming}: Complex behaviors (pain, PTSD,
  fear, sleep necessity) must emerge from the equations, not be
  explicitly coded. If a behavior must be hardcoded, the theory is
  incomplete.
\end{enumerate}

\begin{center}\rule{0.5\linewidth}{0.5pt}\end{center}

\subsection{2. Body Systems: Sensory
Organs}\label{body-systems-sensory-organs}

\subsubsection{2.1 Eye: Visual Impedance
Transduction}\label{eye-visual-impedance-transduction}

The Γ-Net eye is a 16×16 pixel retina with a physics-based visual
processing pipeline:

\paragraph{2.1.1 Architecture}\label{architecture}

{\def\LTcaptype{none} % do not increment counter
\begin{longtable}[]{@{}
  >{\raggedright\arraybackslash}p{(\linewidth - 4\tabcolsep) * \real{0.3333}}
  >{\raggedright\arraybackslash}p{(\linewidth - 4\tabcolsep) * \real{0.3333}}
  >{\raggedright\arraybackslash}p{(\linewidth - 4\tabcolsep) * \real{0.3333}}@{}}
\toprule\noalign{}
\begin{minipage}[b]{\linewidth}\raggedright
Component
\end{minipage} & \begin{minipage}[b]{\linewidth}\raggedright
Function
\end{minipage} & \begin{minipage}[b]{\linewidth}\raggedright
Output
\end{minipage} \\
\midrule\noalign{}
\endhead
\bottomrule\noalign{}
\endlastfoot
Retina (16×16) & Raw light capture & Pixel intensity matrix \\
Saccade Controller & Foveal targeting via Γ-gradient & (x, y) fixation
point \\
Edge Detector & Contrast extraction & Edge map \\
V1 Gabor Filter Bank & Orientation-selective filtering & Feature Γ
vector \\
Impedance Encoder & Feature → Γ conversion & Visual Γ array \\
\end{longtable}
}

\paragraph{2.1.2 Saccade Mechanism}\label{saccade-mechanism}

The eye does not process the entire visual field uniformly. A saccade
controller directs the fovea (high-resolution center) toward regions of
maximum Γ — the areas of greatest mismatch between expected and observed
patterns:

\[\text{saccade\_target} = \arg\max_{(x,y)} \Gamma_{local}(x, y)\]

This implements attention as \textbf{impedance-guided exploration}: the
saccade controller targets regions of maximum Γ, not confirming stimuli.

\paragraph{2.1.3 Edge Detection as Impedance
Boundary}\label{edge-detection-as-impedance-boundary}

Edge detection is reframed as impedance boundary detection. At each
pixel, the local impedance is computed from luminance contrast:

\[\Gamma_{pixel}(x,y) = \frac{I(x,y) - \bar{I}_{neighborhood}}{I(x,y) + \bar{I}_{neighborhood}}\]

where \(I(x,y)\) is pixel intensity and \(\bar{I}_{neighborhood}\) is
the mean intensity of the surrounding region. High \(|\Gamma_{pixel}|\)
indicates a contrast boundary — an edge.

\paragraph{2.1.4 The Retinotopic Map as
Γ-Topology}\label{the-retinotopic-map-as-ux3b3-topology}

The lens of the eye is a physical Fourier transformer: the focal plane
of a convex lens contains the spatial frequency spectrum of the input
light field. Each retinal position corresponds to a spatial frequency,
and nearby positions correspond to nearby frequencies. In impedance
terms, adjacent photoreceptors transduce similar spatial frequencies and
thus produce similar impedance values:

\[d_{retina}(i, j) = \left| \frac{Z_i - Z_j}{Z_i + Z_j} \right| \approx 0 \quad \text{for adjacent } i,j\]

The retinotopic map is therefore a
\textbf{hardware realization of Γ-topology}: the spatial arrangement of
photoreceptors is an impedance gradient frozen into physical structure
by lens optics. Low spatial frequencies (large contours) map to δ/θ
bands and project preferentially to right-hemisphere holistic
processing; high spatial frequencies (texture, detail) map to β/γ bands
and project to left-hemisphere sequential processing. The brain does not
choose what to see — the physics of the Fourier lens determines which
frequencies arrive where.

\paragraph{2.1.5 Experimental
Verification}\label{experimental-verification}

\texttt{exp\_eye\_oscilloscope.py} verified: - Saccade stability:
fixation converges within 3 steps ✅ - Edge detection accuracy: matches
Sobel operator output ✅ - Impedance encoding: visual Γ ∈ {[}-1, 1{]}
for all inputs ✅ - Foveal resolution gradient: center 4× resolution
vs.~periphery ✅

\subsubsection{2.2 Ear: Cochlear Impedance
Analysis}\label{ear-cochlear-impedance-analysis}

\paragraph{2.2.1 Architecture}\label{architecture-1}

The Γ-Net ear implements a 24-band cochlear filterbank based on the
Glasberg-Moore auditory filter model (Glasberg \& Moore, 1990):

{\def\LTcaptype{none} % do not increment counter
\begin{longtable}[]{@{}
  >{\raggedright\arraybackslash}p{(\linewidth - 4\tabcolsep) * \real{0.3333}}
  >{\raggedright\arraybackslash}p{(\linewidth - 4\tabcolsep) * \real{0.3333}}
  >{\raggedright\arraybackslash}p{(\linewidth - 4\tabcolsep) * \real{0.3333}}@{}}
\toprule\noalign{}
\begin{minipage}[b]{\linewidth}\raggedright
Component
\end{minipage} & \begin{minipage}[b]{\linewidth}\raggedright
Function
\end{minipage} & \begin{minipage}[b]{\linewidth}\raggedright
Output
\end{minipage} \\
\midrule\noalign{}
\endhead
\bottomrule\noalign{}
\endlastfoot
Outer Ear & Frequency-dependent gain & Amplified signal \\
Cochlear Filterbank & 24-band ERB decomposition & Tonotopic
activation \\
Hair Cell Transduction & Mechanical → neural conversion & Auditory Γ per
band \\
Tonotopic Map & Frequency → spatial mapping & 24-dim cochlear
fingerprint \\
\end{longtable}
}

\paragraph{2.2.2 Frequency-to-Impedance
Mapping}\label{frequency-to-impedance-mapping}

Each cochlear band has a characteristic impedance determined by its
center frequency:

\[Z_{band}(f) = Z_0 \cdot \left(1 + Q \cdot \left|\frac{f - f_c}{f_c}\right|\right)\]

where \(f_c\) is the center frequency and \(Q\) is the quality factor.
Sounds matching a band’s center frequency produce \(\Gamma \to 0\)
(perfect matching); sounds at distant frequencies produce high
\(\Gamma\) (mismatch).

\paragraph{2.2.3 Cochlear Fingerprints}\label{cochlear-fingerprints}

Each auditory stimulus produces a unique 24-dimensional “cochlear
fingerprint” — the vector of Γ values across all bands. This fingerprint
serves as the fundamental auditory representation:

\[\mathbf{F}_{sound} = [\Gamma_1, \Gamma_2, ..., \Gamma_{24}]\]

Two sounds are perceived as similar when their fingerprint distance is
small:

\[d(A, B) = \|\mathbf{F}_A - \mathbf{F}_B\|_2\]

This replaces spectrogram-based representations with an impedance-based
representation that naturally supports Hebbian association.

\paragraph{2.2.4 The Tonotopic Map as
Γ-Topology}\label{the-tonotopic-map-as-ux3b3-topology}

The basilar membrane of the cochlea is a graded elastic strip: narrow
and stiff at the base (high-frequency resonance), wide and soft at the
apex (low-frequency resonance). This physical gradient creates a
continuous impedance mapping along the membrane:

\[Z_{basilar}(x) = Z_{base} \cdot \left(\frac{L - x}{L}\right)^\alpha\]

where \(x\) is position along the membrane and \(L\) is total length.
Adjacent hair cells resonate at similar frequencies and present similar
impedances, yielding \(\Gamma_{ij} \to 0\) for neighboring cells. The
tonotopic map is therefore a \textbf{hardware realization of
Γ-topology}: the spatial organization of the cochlea is an impedance
gradient frozen into the physical structure of the basilar membrane.
This is not the brain computing frequencies — it is the physics of a
graded elastic strip performing mechanical Fourier decomposition.

Notably, the auditory nerve transmits at 50Ω, while the temporal cortex
receives at 75Ω — an impedance mismatch of
\(\Gamma = |50 - 75|/(50 + 75) = 0.20\). By contrast, the optic nerve
(50Ω) matches the occipital cortex (50Ω) perfectly (\(\Gamma = 0\)).
This asymmetry may explain why auditory processing requires more
temporal calibration than visual processing, and contributes to the
temporal cortex’s specialization in time-domain analysis.

\paragraph{2.2.5 Auditory Grounding}\label{auditory-grounding}

\texttt{exp\_auditory\_grounding.py} verified the hear→learn→recognize
pipeline: - Vowel discrimination: /a/, /i/, /u/ produce distinct
cochlear fingerprints ✅ - Cross-modal binding: auditory Γ + visual Γ →
integrated percept ✅ - Frequency selectivity: Q-factor matches human
psychoacoustic data ✅ - Adaptation: repeated stimuli reduce auditory Γ
(habituation) ✅

\subsubsection{2.3 Hand: Motor Impedance
System}\label{hand-motor-impedance-system}

\paragraph{2.3.1 Architecture}\label{architecture-2}

The Γ-Net hand is a 5-finger system with pressure sensors, temperature
sensors, and motor actuators:

{\def\LTcaptype{none} % do not increment counter
\begin{longtable}[]{@{}
  >{\raggedright\arraybackslash}p{(\linewidth - 4\tabcolsep) * \real{0.3333}}
  >{\raggedright\arraybackslash}p{(\linewidth - 4\tabcolsep) * \real{0.3333}}
  >{\raggedright\arraybackslash}p{(\linewidth - 4\tabcolsep) * \real{0.3333}}@{}}
\toprule\noalign{}
\begin{minipage}[b]{\linewidth}\raggedright
Component
\end{minipage} & \begin{minipage}[b]{\linewidth}\raggedright
Function
\end{minipage} & \begin{minipage}[b]{\linewidth}\raggedright
Γ Mapping
\end{minipage} \\
\midrule\noalign{}
\endhead
\bottomrule\noalign{}
\endlastfoot
5 Pressure Sensors & Contact force detection &
\(\Gamma_p = |F - F_{target}| / (F + F_{target})\) \\
5 Temperature Sensors & Thermal environment &
\(\Gamma_T = |T - T_{comfort}| / (T + T_{comfort})\) \\
Grip Controller & Force calibration & Motor Γ \\
Proprioception & Internal position sense & Calibration Γ \\
\end{longtable}
}

\paragraph{2.3.2 Motor Calibration}\label{motor-calibration}

Grip force follows an impedance calibration curve:

\[F_{grip}(t+1) = F_{grip}(t) - \eta_{motor} \cdot \Gamma_{grip} \cdot \text{sign}(Z_{hand} - Z_{object})\]

where \(Z_{hand}\) is the hand’s current impedance setting and
\(Z_{object}\) is the object’s impedance (hard, soft, fragile). Learning
to grip is learning to match impedance: too much force
(\(Z_{hand} \ll Z_{object}\)) crushes; too little
(\(Z_{hand} \gg Z_{object}\)) drops.

\paragraph{2.3.3 Anxiety-Induced Tremor}\label{anxiety-induced-tremor}

Under high system stress (\(T > T_{tremor}\)), noise is injected into
motor impedance:

\[Z_{hand}(t) = Z_{hand,0} + A_{tremor} \cdot \text{noise}(t) \cdot T / T_{max}\]

This produces the anxiety tremor observed clinically — hands shake not
because the motor system is damaged but because system-wide Γ elevation
introduces noise into motor channels.

\paragraph{2.3.4 Experimental
Verification}\label{experimental-verification-1}

\texttt{exp\_hand\_coordination.py} verified five clinical scenarios: -
Normal grip calibration: converges within 10 trials ✅ - Fragile object
handling: reduced force + increased caution ✅ - Anxiety tremor: tremor
amplitude proportional to stress level ✅ - Motor learning transfer:
skills generalize across similar objects ✅ - Fatigue: prolonged
gripping increases Γ\_motor (muscle fatigue analog) ✅

\subsubsection{2.4 Mouth: Articulatory
System}\label{mouth-articulatory-system}

\paragraph{2.4.1 Architecture}\label{architecture-3}

The Γ-Net mouth produces speech through impedance-based articulatory
planning:

{\def\LTcaptype{none} % do not increment counter
\begin{longtable}[]{@{}
  >{\raggedright\arraybackslash}p{(\linewidth - 4\tabcolsep) * \real{0.3333}}
  >{\raggedright\arraybackslash}p{(\linewidth - 4\tabcolsep) * \real{0.3333}}
  >{\raggedright\arraybackslash}p{(\linewidth - 4\tabcolsep) * \real{0.3333}}@{}}
\toprule\noalign{}
\begin{minipage}[b]{\linewidth}\raggedright
Component
\end{minipage} & \begin{minipage}[b]{\linewidth}\raggedright
Function
\end{minipage} & \begin{minipage}[b]{\linewidth}\raggedright
Γ Mapping
\end{minipage} \\
\midrule\noalign{}
\endhead
\bottomrule\noalign{}
\endlastfoot
Broca’s Planner & Articulatory sequencing & \(\Gamma_{plan}\) \\
Vocal Tract Model & Formant-based vowel generation & F1, F2
parameters \\
Motor Execution & Plan → sound conversion & \(\Gamma_{speech}\) \\
Auditory Feedback & Self-monitoring via ear & \(\Gamma_{feedback}\) \\
\end{longtable}
}

\paragraph{2.4.2 Vowel Production}\label{vowel-production}

Five vowels (/a/, /e/, /i/, /o/, /u/) are defined by formant frequency
pairs (F1, F2), following Peterson \& Barney (1952):

{\def\LTcaptype{none} % do not increment counter
\begin{longtable}[]{@{}llll@{}}
\toprule\noalign{}
Vowel & F1 (Hz) & F2 (Hz) & Articulatory Description \\
\midrule\noalign{}
\endhead
\bottomrule\noalign{}
\endlastfoot
/a/ & 730 & 1090 & Open, back \\
/e/ & 530 & 1840 & Mid, front \\
/i/ & 270 & 2290 & Close, front \\
/o/ & 570 & 840 & Mid, back \\
/u/ & 300 & 870 & Close, back \\
\end{longtable}
}

Speaking is impedance matching between the internal concept
representation and the external acoustic output:

\[\Gamma_{speech} = \frac{|\mathbf{F}_{produced} - \mathbf{F}_{intended}|}{|\mathbf{F}_{produced} + \mathbf{F}_{intended}|}\]

Practice reduces \(\Gamma_{speech}\) — repeated articulation
monotonically decreases the speech impedance mismatch.

\paragraph{2.4.3 Sensorimotor Loop}\label{sensorimotor-loop}

The mouth implements a complete sensorimotor loop (Hickok \& Poeppel,
2007):

\begin{enumerate}
\def\labelenumi{\arabic{enumi}.}
\tightlist
\item
  \textbf{Broca}: Plans articulatory sequence from concept activation
\item
  \textbf{Vocal Tract}: Executes plan → produces acoustic signal
\item
  \textbf{Ear}: Receives self-generated sound (auditory feedback)
\item
  \textbf{Wernicke}: Processes heard speech → generates cochlear
  fingerprint
\item
  \textbf{Error signal}:
  \(\Delta\Gamma = \Gamma_{heard} - \Gamma_{intended}\)
\item
  \textbf{Broca update}: Adjusts plan to reduce error
\end{enumerate}

This is the physics of \textbf{learning to talk}: infants babble (random
motor plans), hear themselves (auditory feedback), and gradually
calibrate Broca’s impedance settings to produce intended sounds.

\begin{center}\rule{0.5\linewidth}{0.5pt}\end{center}

\subsection{3. Brain Systems}\label{brain-systems}

\subsubsection{3.1 FusionBrain: The Integration
Engine}\label{fusionbrain-the-integration-engine}

The FusionBrain is the central integration module that combines all
sensory inputs, internal states, and cognitive outputs into a unified Γ
map:

\begin{Shaded}
\begin{Highlighting}[]
\KeywordTok{class}\NormalTok{ FusionBrain:}
    \CommentTok{"""}
\CommentTok{    Fusion Brain: v3 neural substrate + v4 communication protocol}

\CommentTok{    Unifies 4 brain regions, Γ{-}Net v4 messaging protocol, and performance analytics.}
\CommentTok{    Supports the complete stimulus → cognition → emotion → motor → memory cycle.}
\CommentTok{    """}
\end{Highlighting}
\end{Shaded}

\paragraph{3.1.1 Sensory Fusion}\label{sensory-fusion}

Cross-modal binding is computed as Γ correlation across modalities:

\[\text{binding}_{AB} = 1 - |\Gamma_A - \Gamma_B|\]

When visual and auditory Γ values are similar (both low or both high),
binding is strong — the FusionBrain module outputs a bound
classification. When they diverge, binding weakens — the module outputs
separate events. This replaces the “binding problem” with impedance
matching.

\paragraph{3.1.2 Arousal Computation}\label{arousal-computation}

System arousal is the mean reflected energy across all active channels:

\[\text{arousal} = \frac{1}{N}\sum_{i=1}^{N} \Gamma_i^2\]

High arousal (many channels mismatched) triggers sympathetic activation;
low arousal (channels well-matched) permits parasympathetic rest.

\subsubsection{3.2 LifeLoop: The Self-Sustaining
Cycle}\label{lifeloop-the-self-sustaining-cycle}

The LifeLoop is the master control loop that keeps ALICE alive:

\begin{verbatim}
while alive:
    signals = perceive()              # Multi-modal sensory input
    errors  = estimate_errors()       # Cross-modal error estimation
    commands = compensate(errors)      # Generate motor compensation commands
    execute(commands)                  # Body execution
    feedback = re_perceive()           # Action changes perception
    calibrate(feedback)               # Update calibration parameters
    adapt(performance)                # Meta-learning adjusts the system
\end{verbatim}

\texttt{exp\_life\_loop.py} verified continuous autonomous operation
over 1000+ ticks with no NaN, no Inf, and stable vital signs ✅.

\subsubsection{3.3 Pain and Nociception}\label{pain-and-nociception}

\paragraph{3.3.1 Pain as Impedance Mismatch
Energy}\label{pain-as-impedance-mismatch-energy}

Pain is not a dedicated signal but the \textbf{energy cost of impedance
mismatch} (cumulative reflected energy):

\[E_{\text{ref}} = \sum_{i \in \text{nociceptive}} \Gamma_i^2 \cdot w_i\]

where \(w_i\) are channel-specific weights (thermal channels weigh more
than proprioceptive channels, matching clinical pain sensitivity
distributions). Anomalous peaks in \(E_{\text{ref}}\) constitute the
pain correlate — the physical quantity whose phenomenological
interpretation is nociceptive experience.

\paragraph{3.3.2 The Pain-Consciousness
Loop}\label{the-pain-consciousness-loop}

Pain and consciousness are coupled through a critical feedback loop:

\[\mathcal{C}_{\Gamma,t+1} = f(\mathcal{C}_{\Gamma,t}, E_{\text{ref},t}, \Theta_t)\]

\begin{itemize}
\tightlist
\item
  Moderate \(E_{\text{ref}}\) (\(\Gamma \in [0.3, 0.7]\)): Coherence
  increases (alerting function)
\item
  Severe \(E_{\text{ref}}\) (\(\Gamma > 0.9\)): Coherence collapses
  (dissociative protection)
\item
  Chronic \(E_{\text{ref}}\): Arousal \(\Theta\) rises → cooling fails →
  impedance-locked state (PTSD)
\end{itemize}

\texttt{exp\_pain\_collapse.py} verified the \(E_{\text{ref}}\) collapse
curve — the exact Γ threshold at which coherence transitions from
alerting to collapsing ✅.

\paragraph{3.3.3 Pain Sensitization}\label{pain-sensitization}

Repeated pain exposure modifies channel impedance permanently:

\[Z_{pain,i}(t) = Z_{pain,i}(0) \cdot (1 + \alpha_{sensitization} \cdot n_{exposures})\]

This implements pain sensitization: each trauma episode increases the
channel’s baseline impedance, making it more reactive to future stimuli.
The sensitization factor (\(\alpha = 0.005\)) is irreversible —
consistent with the clinical observation that PTSD patients have
permanently lowered pain thresholds.

\subsubsection{3.4 Autonomic Nervous
System}\label{autonomic-nervous-system}

\paragraph{3.4.1 Dual-Branch
Architecture}\label{dual-branch-architecture}

The autonomic system has two branches that modulate physiological
responses based on system-wide Γ:

\textbf{Sympathetic Branch} (fight-or-flight):
\[S(t) = S_{base} + \alpha_S \cdot \text{mean}(\Gamma_{threat}^2) + \beta_S \cdot \text{Pain}\]

\textbf{Parasympathetic Branch} (rest-and-digest):
\[P(t) = P_{base} + \alpha_P \cdot (1 - \text{arousal}) - \beta_P \cdot \text{Pain}\]

\paragraph{3.4.2 Vital Signs}\label{vital-signs}

The autonomic system generates four vital signs:

{\def\LTcaptype{none} % do not increment counter
\begin{longtable}[]{@{}lll@{}}
\toprule\noalign{}
Vital Sign & Formula & Normal Range \\
\midrule\noalign{}
\endhead
\bottomrule\noalign{}
\endlastfoot
Heart Rate & \(HR = 60 + 40 \cdot S - 20 \cdot P\) & 55–100 bpm \\
Cortisol & \(C = 0.1 + 0.4 \cdot S\) & 0.1–0.5 \\
Temperature & \(T = 36.5 + 1.5 \cdot \text{arousal}\) & 36.5–38.0°C \\
Respiration & \(R = 12 + 8 \cdot S - 4 \cdot P\) & 10–20 /min \\
\end{longtable}
}

All vital signs are direct consequences of Γ dynamics — no separate
“vital sign generator” exists.

\subsubsection{3.5 Sleep Physics}\label{sleep-physics}

\paragraph{3.5.1 Sleep as Offline Impedance
Restructuring}\label{sleep-as-offline-impedance-restructuring}

Sleep is not a functional pause but a \textbf{physically necessary mode
of operation}. During waking, the system minimizes external impedance
mismatch (\(\Gamma_{external}\)). During sleep, it minimizes internal
impedance mismatch (\(\Gamma_{internal}\)):

\begin{itemize}
\tightlist
\item
  \textbf{Waking}: \(\min \Gamma_{ext}\) (match the world)
\item
  \textbf{Sleeping}: \(\min \Gamma_{int}\) (repair yourself)
\end{itemize}

\paragraph{3.5.2 Sleep Stages}\label{sleep-stages}

Γ-Net implements four biologically inspired sleep stages:

{\def\LTcaptype{none} % do not increment counter
\begin{longtable}[]{@{}llll@{}}
\toprule\noalign{}
Stage & Duration & Function & Γ Operation \\
\midrule\noalign{}
\endhead
\bottomrule\noalign{}
\endlastfoot
N1 (Light) & 5\% & Transition & Gradual sensory Γ gating \\
N2 (Spindle) & 50\% & Memory selection & K-complex Γ filtering \\
N3 (Slow-wave) & 25\% & Deep repair & Global Γ normalization \\
REM (Dream) & 20\% & Memory testing & Simulated Γ activation \\
\end{longtable}
}

\paragraph{3.5.3 Sleep Necessity Proof}\label{sleep-necessity-proof}

\texttt{exp\_sleep\_physics.py} provided a direct experimental proof
that sleep is physically necessary:

\begin{enumerate}
\def\labelenumi{\arabic{enumi}.}
\item
  \textbf{Sleep deprivation}: After 200 ticks without sleep, mean Γ
  increases by 40\%, consciousness Φ decreases by 60\%, and motor
  coordination degrades by 50\%.
\item
  \textbf{Recovery sleep}: A single complete sleep cycle (N1→N2→N3→REM)
  restores all metrics to baseline within 5\% tolerance.
\item
  \textbf{Sleep debt accumulation}: Sleep pressure follows
  adenosine-like kinetics — it accumulates during waking and can only be
  discharged during N3 sleep.
\item
  \textbf{Insomnia paradox}: PTSD-frozen states prevent sleep entry
  (consciousness \textless{} 0.15 blocks sleep\_cycle.tick()), creating
  a vicious cycle: frozen → can’t sleep → can’t repair → stays frozen.
\end{enumerate}

\paragraph{3.5.4 Circadian Regulation}\label{circadian-regulation}

\texttt{exp\_day\_night\_cycle.py} verified a complete 1440-tick
(24-hour) circadian cycle:

\begin{itemize}
\tightlist
\item
  Dawn (tick 0–360): Cortisol surge, wake transition, consciousness
  rising
\item
  Day (tick 360–1080): Active perception, learning, stress accumulation
\item
  Dusk (tick 1080–1200): Melatonin-analog rise, arousal decline
\item
  Night (tick 1200–1440): 6 complete sleep cycles, impedance repair
\end{itemize}

\subsubsection{3.6 Memory Hierarchy}\label{memory-hierarchy}

\paragraph{3.6.1 Working Memory}\label{working-memory}

Working memory implements Miller’s Law (Miller, 1956) through an
impedance-gated buffer:

\begin{itemize}
\tightlist
\item
  \textbf{Capacity}: 7 ± 2 items (verified in HIP stress test: capacity
  = 7 ✅)
\item
  \textbf{Encoding gate}: Item enters only if attention Γ \textless{}
  θ\_encode
\item
  \textbf{Decay}: Items decay at rate \(\lambda_{WM} = 0.05\)/tick
  (approximately 14 ticks half-life)
\item
  \textbf{Refresh}: Active rehearsal resets decay timer
\item
  \textbf{Overflow}: When capacity is exceeded, the item with highest Γ
  (least matching) is evicted
\end{itemize}

\paragraph{3.6.2 Hippocampus}\label{hippocampus}

The hippocampus stores episodic memories with impedance-modulated decay
(Scoville \& Milner, 1957; Tulving, 1972):

\begin{itemize}
\tightlist
\item
  \textbf{Capacity}: 1000 episodes (LRU eviction when full)
\item
  \textbf{Encoding}: Triggered by novelty
  (\(\Gamma_{novelty} > \theta_{encode}\))
\item
  \textbf{Decay rate}:
  \(\lambda_{eff} = \lambda_{base} / (1 - \Gamma^2)\) (Equation 2 from
  Paper I)
\item
  \textbf{Retrieval}: Cue-based search via Γ similarity matching
\item
  \textbf{Consolidation}: During N3 sleep, high-T episodes transfer to
  semantic field
\end{itemize}

\paragraph{3.6.3 Semantic Field}\label{semantic-field}

The semantic field stores concepts as impedance patterns:

\begin{itemize}
\tightlist
\item
  \textbf{Representation}: Each concept has a mass (\(m\)), frequency
  (\(f\)), Γ value, and set of associations
\item
  \textbf{Learning}: Hebbian reinforcement — co-activated concepts
  reduce mutual Γ
\item
  \textbf{Grounding}: Concepts are bound to sensory fingerprints via
  cross-modal Γ matching
\item
  \textbf{Capacity}: Effectively unlimited (concepts are not stored
  discretely but as impedance modifications to a continuous field)
\end{itemize}

\paragraph{3.6.4 Memory Verification}\label{memory-verification}

\texttt{exp\_memory\_theory.py} verified all 5 memory properties:

{\def\LTcaptype{none} % do not increment counter
\begin{longtable}[]{@{}
  >{\raggedright\arraybackslash}p{(\linewidth - 6\tabcolsep) * \real{0.2500}}
  >{\raggedright\arraybackslash}p{(\linewidth - 6\tabcolsep) * \real{0.2500}}
  >{\raggedright\arraybackslash}p{(\linewidth - 6\tabcolsep) * \real{0.2500}}
  >{\raggedright\arraybackslash}p{(\linewidth - 6\tabcolsep) * \real{0.2500}}@{}}
\toprule\noalign{}
\begin{minipage}[b]{\linewidth}\raggedright
\#
\end{minipage} & \begin{minipage}[b]{\linewidth}\raggedright
Property
\end{minipage} & \begin{minipage}[b]{\linewidth}\raggedright
Verification
\end{minipage} & \begin{minipage}[b]{\linewidth}\raggedright
Status
\end{minipage} \\
\midrule\noalign{}
\endhead
\bottomrule\noalign{}
\endlastfoot
1 & Ebbinghaus forgetting curve & Decay matches exponential with
impedance modulation & ✅ \\
2 & Flashbulb memory & High-Γ memories persist indefinitely & ✅ \\
3 & Working memory capacity & 7 ± 2 items, Miller’s Law & ✅ \\
4 & Sleep consolidation & N3 transfers episodic → semantic & ✅ \\
5 & Retrieval failure & High-interference Γ blocks cue matching & ✅ \\
\end{longtable}
}

\subsubsection{3.7 Thalamus: The Sensory
Gate}\label{thalamus-the-sensory-gate}

The thalamus serves as the central gating mechanism (Sherman \&
Guillery, 2006; Crick, 1984):

\[\text{pass}(x) = \begin{cases} x & \text{if } \Gamma_x < \theta_{gate} \text{ or } x \in \text{attention\_set} \\ 0 & \text{otherwise} \end{cases}\]

\begin{itemize}
\tightlist
\item
  \textbf{Bottom-up gating}: Only stimuli with sufficient salience
  (\(\Gamma > \theta_{salience}\)) reach cortex
\item
  \textbf{Top-down modulation}: PFC sends attention bias to thalamus,
  lowering thresholds for goal-relevant stimuli
\item
  \textbf{Reticular nucleus}: Implements the “searchlight” (Crick, 1984)
  — a roving attention spotlight that scans channels sequentially
\end{itemize}

\texttt{exp\_thalamus\_amygdala.py} verified thalamic gating, amygdala
fear conditioning, and their interaction (8/8 experiments passed) ✅.

\subsubsection{3.8 Amygdala: Fear and Emotional
Tagging}\label{amygdala-fear-and-emotional-tagging}

The amygdala implements Pavlovian fear conditioning (Pavlov, 1927;
LeDoux, 1996):

\[\Gamma_{fear}(CS) = \Gamma_{US} \cdot (1 - \alpha)^{n_{pairings}} + \Gamma_{CS,0}\]

After as few as 3–5 CS-US pairings, the CS acquires the fear-inducing
impedance characteristics of the US. Extinction reduces but never
eliminates this association:

\[\Gamma_{fear,extinction} = \Gamma_{fear} \cdot (1 - \beta_{extinction})^{n_{extinction}} + \Gamma_{residual}\]

where \(\Gamma_{residual} > 0\) always — fear memories can be suppressed
but never erased. This matches the clinical reality of anxiety
disorders.

\subsubsection{3.9 Basal Ganglia: Action
Selection}\label{basal-ganglia-action-selection}

The basal ganglia selects actions based on transmission efficiency:

\[P(a) = \frac{e^{T_a / \tau}}{\sum_j e^{T_j / \tau}}\]

where \(T_a = 1 - \Gamma_a^2\) is the transmission efficiency of action
\(a\) and \(\tau\) is a temperature parameter (not to be confused with
system temperature). High-efficiency actions are selected more often;
low-efficiency actions are explored occasionally. Dopamine modulates
\(\tau\): high dopamine → more exploitation; low dopamine → more
exploration.

\subsubsection{3.10 Prefrontal Cortex: Executive
Control}\label{prefrontal-cortex-executive-control}

The PFC implements executive functions with finite cognitive energy:

\begin{itemize}
\tightlist
\item
  \textbf{Goal maintenance}: Holds current goal in a dedicated working
  memory slot
\item
  \textbf{Task switching}: Changes goal when current goal Γ exceeds
  frustration threshold
\item
  \textbf{Inhibitory control}: Suppresses prepotent responses (high-T
  actions that conflict with goal)
\item
  \textbf{Energy depletion}: Each executive operation costs energy; when
  energy is depleted, the system falls back to habitual (basal ganglia)
  control — this is the physics of ego depletion
\end{itemize}

\texttt{exp\_prefrontal.py} and \texttt{exp\_cognitive\_flexibility.py}
verified: - Task switching cost \textless{} 300ms equivalent ✅ -
Perseveration \textless{} 30\% on Wisconsin Card Sort analog ✅ - Energy
depletion → increased impulsivity ✅ - PFC recovery via sleep ✅

\subsubsection{3.11 Hippocampus-Wernicke
Integration}\label{hippocampus-wernicke-integration}

The hippocampus and Wernicke’s area form an integrated memory-language
system:

\begin{enumerate}
\def\labelenumi{\arabic{enumi}.}
\tightlist
\item
  \textbf{Hippocampus} records episodes with contextual Γ values
\item
  \textbf{Wernicke’s area} processes language input (heard speech) via
  sequential prediction
\item
  \textbf{Integration}: Wernicke’s temporal predictions generate chunks;
  chunks enter hippocampus as episodes; during sleep, episodes
  consolidate to semantic field; semantic field influences Wernicke’s
  predictions
\end{enumerate}

\texttt{exp\_episodic\_wernicke.py} verified 8 integration properties: -
Wernicke sequential prediction generates N400-like Γ anomaly for
unexpected words ✅ - Hippocampal encoding triggered by novelty Γ ✅ -
Chunk formation from temporal co-occurrence ✅ - Sleep consolidation of
chunks to semantic field ✅

\subsubsection{3.12 Consciousness Module}\label{consciousness-module}

\paragraph{3.12.1 Φ Computation}\label{ux3c6-computation}

Consciousness level is computed at each tick as:

\[\Phi = f\left(\bar{T}, \text{arousal}, \text{binding}, \text{wakefulness}\right)\]

where \(\bar{T} = \frac{1}{N}\sum_i (1 - \Gamma_i^2)\) is the mean
transmission efficiency.

\paragraph{3.12.2 Consciousness States}\label{consciousness-states}

{\def\LTcaptype{none} % do not increment counter
\begin{longtable}[]{@{}lll@{}}
\toprule\noalign{}
State & Φ Range & Condition \\
\midrule\noalign{}
\endhead
\bottomrule\noalign{}
\endlastfoot
Deep coma & 0.00–0.05 & Massive Γ → 1 across all channels \\
Vegetative & 0.05–0.15 & Some reflexive channels active \\
Minimal & 0.15–0.30 & Sparse conscious access \\
Drowsy & 0.30–0.50 & Reduced integration \\
Alert & 0.50–0.80 & Normal waking consciousness \\
Hyperfocused & 0.80–1.00 & Peak performance / flow state \\
\end{longtable}
}

\paragraph{3.12.3 Consciousness
Flickering}\label{consciousness-flickering}

Post-trauma, consciousness does not simply “switch off” — it flickers
between states as competing channel dynamics push Φ up and down.
\texttt{exp\_awakening.py} captured this dynamic: Φ ranged from 0.000
(post-trauma collapse) to 0.777 (alert waking), with the transition
exhibiting chaotic fluctuations rather than smooth recovery ✅.

\subsubsection{3.13 Additional Brain Modules
Summary}\label{additional-brain-modules-summary}

The following modules each implement specific cognitive functions within
the unified Γ framework:

{\def\LTcaptype{none} % do not increment counter
\begin{longtable}[]{@{}
  >{\raggedright\arraybackslash}p{(\linewidth - 4\tabcolsep) * \real{0.3333}}
  >{\raggedright\arraybackslash}p{(\linewidth - 4\tabcolsep) * \real{0.3333}}
  >{\raggedright\arraybackslash}p{(\linewidth - 4\tabcolsep) * \real{0.3333}}@{}}
\toprule\noalign{}
\begin{minipage}[b]{\linewidth}\raggedright
Module
\end{minipage} & \begin{minipage}[b]{\linewidth}\raggedright
Function
\end{minipage} & \begin{minipage}[b]{\linewidth}\raggedright
Key Γ Interaction
\end{minipage} \\
\midrule\noalign{}
\endhead
\bottomrule\noalign{}
\endlastfoot
\texttt{mirror\_neurons.py} & Empathy \& motor imitation &
\(\Gamma_{mirror} = \Gamma_{observed}\) (automatic copying) \\
\texttt{curiosity\_drive.py} & Novelty seeking & Curiosity ∝ prediction
error Γ \\
\texttt{attention\_plasticity.py} & Adaptive attention & Training
reduces attention Γ \\
\texttt{impedance\_adaptation.py} & System-wide calibration & Global Γ
optimization daemon \\
\texttt{sleep\_physics.py} & Sleep + impedance debt tracking & Debt
accumulates, sleep repairs \\
\texttt{calibration.py} & Developmental calibration + dynamic time slice
& Birth → adult Γ trajectory; adaptive processing speed \\
\texttt{emotion\_granularity.py} & Fine-grained affect & Emotion vector
in Γ space \\
\texttt{thalamus.py} & Sensory gating + arousal modulation &
Yerkes-Dodson arousal curve; optimal Γ for performance \\
\end{longtable}
}

\begin{center}\rule{0.5\linewidth}{0.5pt}\end{center}

\subsection{4. The Perception Pipeline}\label{the-perception-pipeline}

\subsubsection{4.1 Complete 15-Step
Pipeline}\label{complete-15-step-pipeline}

Every tick, ALICE executes the following perception pipeline in strict
order:

{\def\LTcaptype{none} % do not increment counter
\begin{longtable}[]{@{}llll@{}}
\toprule\noalign{}
Step & Module & Operation & Complexity \\
\midrule\noalign{}
\endhead
\bottomrule\noalign{}
\endlastfoot
1 & Eye & \texttt{see()} → visual Γ & O(1) \\
2 & Ear & \texttt{hear()} → auditory Γ & O(1) \\
3a & Thalamus & Bottom-up gating & O(1) \\
3b & PFC → Thalamus & Top-down attention bias & O(1) \\
4 & Working Memory & Update buffer & O(1) \\
5 & FusionBrain & Cross-modal binding & O(1) \\
6 & Nociception & Pain computation & O(1) \\
7 & Emotion & Valence/arousal update & O(1) \\
8 & Autonomic & Vital sign update & O(1) \\
9 & Hippocampus & Episode encoding & O(1) \\
10 & Amygdala & Fear conditioning check & O(1) \\
11 & Basal Ganglia & Action selection & O(1) \\
12a & Broca/Wernicke & Language processing & O(1) \\
12b & Social Resonance & Empathy/ToM tick & O(1) \\
12c & Narrative Memory & Episode weaving & O(1) \\
12d & Recursive Grammar & Grammar tick & O(1) \\
12e & Semantic Pressure & Pressure accumulation & O(1) \\
13 & PFC & Executive monitoring & O(1) \\
14 & Consciousness & \(\mathcal{C}_\Gamma\) coherence & O(1) \\
15 & Sleep Check & Sleep pressure evaluation & O(1) \\
\end{longtable}
}

\textbf{Total pipeline complexity}: O(1) — constant time regardless of
input size.

\subsubsection{4.2 The Impedance-Locked Attractor
State}\label{the-impedance-locked-attractor-state}

A critical architectural feature is the \textbf{impedance-locked
attractor} at the top of \texttt{perceive()}:

\begin{Shaded}
\begin{Highlighting}[]
\CommentTok{\# SystemState.is\_frozen():}
\KeywordTok{def}\NormalTok{ is\_frozen(}\VariableTok{self}\NormalTok{) }\OperatorTok{{-}\textgreater{}} \BuiltInTok{bool}\NormalTok{:}
    \ControlFlowTok{return} \VariableTok{self}\NormalTok{.consciousness }\OperatorTok{\textless{}} \FloatTok{0.15}

\CommentTok{\# AliceBrain.perceive() — only CRITICAL priority can penetrate:}
\ControlFlowTok{if} \VariableTok{self}\NormalTok{.vitals.is\_frozen() }\KeywordTok{and}\NormalTok{ priority }\OperatorTok{!=}\NormalTok{ Priority.CRITICAL:}
    \VariableTok{self}\NormalTok{.\_log\_event(}\StringTok{"perceive\_blocked"}\NormalTok{, \{}
        \StringTok{"reason"}\NormalTok{: }\StringTok{"SYSTEM FROZEN — consciousness too low, only CRITICAL allowed"}\NormalTok{,}
        \StringTok{"consciousness"}\NormalTok{: }\VariableTok{self}\NormalTok{.vitals.consciousness,}
        \StringTok{"pain\_level"}\NormalTok{: }\VariableTok{self}\NormalTok{.vitals.pain\_level,}
\NormalTok{    \})}
    \VariableTok{self}\NormalTok{.vitals.tick(...)  }\CommentTok{\# Still update tick (let the system cool down naturally)}
    \VariableTok{self}\NormalTok{.\_state }\OperatorTok{=} \StringTok{"frozen"}
    \ControlFlowTok{return}\NormalTok{ \{}\StringTok{"status"}\NormalTok{: }\StringTok{"FROZEN"}\NormalTok{, }\StringTok{"vitals"}\NormalTok{: }\VariableTok{self}\NormalTok{.vitals.get\_vitals()\}}
\end{Highlighting}
\end{Shaded}

When consciousness drops below 0.15, the system enters a frozen state.
Non-CRITICAL signals are blocked from progressing through the pipeline —
only CRITICAL-priority stimuli can penetrate. This is the mechanism of
PTSD freezing: the impedance-locked attractor blocks the processing
pipeline required for recovery, but leaves a narrow emergency channel
open.

\subsubsection{4.3 Pipeline Verification}\label{pipeline-verification}

\texttt{exp\_perception\_pipeline.py} verified: - All 15 steps execute
sequentially ✅ - Total pipeline time \textless{} 1ms ✅ - No step
depends on input size ✅ - Impedance-locked attractor activates at
correct thresholds ✅ - Pipeline is idempotent for identical inputs ✅

\begin{center}\rule{0.5\linewidth}{0.5pt}\end{center}

\subsection{5. Communication Protocol}\label{communication-protocol}

\subsubsection{5.1 API Architecture}\label{api-architecture}

Γ-Net ALICE exposes a RESTful API and WebSocket interface for external
interaction:

{\def\LTcaptype{none} % do not increment counter
\begin{longtable}[]{@{}lll@{}}
\toprule\noalign{}
Endpoint & Method & Function \\
\midrule\noalign{}
\endhead
\bottomrule\noalign{}
\endlastfoot
\texttt{/perceive} & POST & Send sensory stimulus \\
\texttt{/state} & GET & Read current brain state \\
\texttt{/vitals} & GET & Read vital signs \\
\texttt{/memory} & GET & Query memory contents \\
\texttt{/speak} & GET & Read latest utterance \\
\texttt{/ws} & WebSocket & Real-time state streaming \\
\texttt{/dashboard} & GET & Web-based visualization \\
\end{longtable}
}

\subsubsection{5.2 State Representation}\label{state-representation}

The system state is serialized as a JSON object containing:

\begin{Shaded}
\begin{Highlighting}[]
\FunctionTok{\{}
  \DataTypeTok{"ram\_temperature"}\FunctionTok{:} \FloatTok{0.0}\FunctionTok{,}
  \DataTypeTok{"stability\_index"}\FunctionTok{:} \FloatTok{1.0}\FunctionTok{,}
  \DataTypeTok{"heart\_rate"}\FunctionTok{:} \FloatTok{72.0}\FunctionTok{,}
  \DataTypeTok{"pain\_level"}\FunctionTok{:} \FloatTok{0.0}\FunctionTok{,}
  \DataTypeTok{"consciousness"}\FunctionTok{:} \FloatTok{0.72}\FunctionTok{,}
  \DataTypeTok{"throttle\_factor"}\FunctionTok{:} \FloatTok{1.0}\FunctionTok{,}
  \DataTypeTok{"is\_frozen"}\FunctionTok{:} \KeywordTok{false}\FunctionTok{,}
  \DataTypeTok{"pain\_events"}\FunctionTok{:} \DecValTok{0}\FunctionTok{,}
  \DataTypeTok{"freeze\_events"}\FunctionTok{:} \DecValTok{0}\FunctionTok{,}
  \DataTypeTok{"recovery\_events"}\FunctionTok{:} \DecValTok{0}\FunctionTok{,}
  \DataTypeTok{"total\_ticks"}\FunctionTok{:} \DecValTok{12345}\FunctionTok{,}
  \DataTypeTok{"pain\_sensitivity"}\FunctionTok{:} \FloatTok{1.0}\FunctionTok{,}
  \DataTypeTok{"baseline\_temperature"}\FunctionTok{:} \FloatTok{0.0}\FunctionTok{,}
  \DataTypeTok{"trauma\_count"}\FunctionTok{:} \DecValTok{0}
\FunctionTok{\}}
\end{Highlighting}
\end{Shaded}

\begin{center}\rule{0.5\linewidth}{0.5pt}\end{center}

\subsection{6. Performance and Stress
Tests}\label{performance-and-stress-tests}

\subsubsection{6.1 System-Wide Metrics}\label{system-wide-metrics}

{\def\LTcaptype{none} % do not increment counter
\begin{longtable}[]{@{}ll@{}}
\toprule\noalign{}
Metric & Value \\
\midrule\noalign{}
\endhead
\bottomrule\noalign{}
\endlastfoot
Source files & 146 \\
Total lines of code & 84,500+ \\
Brain modules & 44 \\
Body organs & 5 \\
Error-correction loops & 7 \\
Independent tests & 1,876 \\
Test pass rate & 100\% \\
Perception complexity & O(1) \\
\end{longtable}
}

\subsubsection{6.2 Integration Stress
Test}\label{integration-stress-test}

A 600-tick stress test conducted during Phase 18 verified system
stability under extreme conditions:

{\def\LTcaptype{none} % do not increment counter
\begin{longtable}[]{@{}
  >{\raggedright\arraybackslash}p{(\linewidth - 4\tabcolsep) * \real{0.3333}}
  >{\raggedright\arraybackslash}p{(\linewidth - 4\tabcolsep) * \real{0.3333}}
  >{\raggedright\arraybackslash}p{(\linewidth - 4\tabcolsep) * \real{0.3333}}@{}}
\toprule\noalign{}
\begin{minipage}[b]{\linewidth}\raggedright
\#
\end{minipage} & \begin{minipage}[b]{\linewidth}\raggedright
Test
\end{minipage} & \begin{minipage}[b]{\linewidth}\raggedright
Result
\end{minipage} \\
\midrule\noalign{}
\endhead
\bottomrule\noalign{}
\endlastfoot
1 & 600-tick continuous operation & No NaN/Inf ✅ \\
2 & PFC depletion marathon & Depletion → recovery ✅ \\
3 & 10 consecutive pain storms & Meltdown → auto-recovery ✅ \\
4 & 200-tick rumination pressure & ≤ 50 cap maintained ✅ \\
5 & No tick \textgreater{} 2 seconds & No deadlock ✅ \\
6 & Rapid calm↔crisis oscillation & Emergency reset → recovery ✅ \\
7 & Full orchestra 600-tick & All subsystems online ✅ \\
8 & Memory stress test & Working memory capacity cap maintained ✅ \\
9 & 5 trauma cascades & Sensitization (2.0×) but no permanent collapse
✅ \\
10 & Clinical grand inspection & 29+ subsystems valid + metacognition
healthy ✅ \\
\end{longtable}
}

\begin{center}\rule{0.5\linewidth}{0.5pt}\end{center}

\subsection{7. Discussion}\label{discussion}

\subsubsection{7.1 Embodiment is Not
Optional}\label{embodiment-is-not-optional}

A common criticism of theories of cognition is that they are
“disembodied” — operating on abstract symbols without physical
grounding. Γ-Net ALICE directly addresses this by deriving all cognition
from physical transduction:

\begin{itemize}
\tightlist
\item
  Vision is impedance boundary detection
\item
  Hearing is frequency-domain impedance analysis
\item
  Touch is force impedance calibration
\item
  Speech is impedance matching between internal and external
  representations
\item
  Pain is the energetic cost of impedance mismatch
\item
  Sleep is offline impedance restructuring
\end{itemize}

Every cognitive operation has a physical interpretation, and every
physical operation has a cognitive consequence.

\subsubsection{7.2 O(1) Perception Matters}\label{o1-perception-matters}

The O(1) perception pipeline is not merely a computational convenience —
it is a \textbf{theoretical commitment}. We argue that biological
perception is inherently O(1): the time from photon hitting retina to
conscious visual percept is approximately constant
(\textasciitilde100ms), regardless of how many objects are in the visual
field. Any adequate theory of perception must account for this
constant-time biological fact.

Γ-Net achieves O(1) through impedance gating: the thalamus pre-filters
inputs by Γ magnitude, ensuring that only a bounded number of signals
reach higher processing — regardless of total input volume.

\subsubsection{7.3 Dynamic Time Slice
Adaptation}\label{dynamic-time-slice-adaptation}

The \texttt{calibration.py} module (class \texttt{TemporalCalibrator})
implements adaptive processing speed — when cognitive load is high (many
channels at high Γ), the system allocates more processing time per tick:

\[\text{time\_slice} = \text{base\_slice} \times (1 + \alpha \cdot \bar{\Gamma}^2)\]

This implements the subjective experience of “time slowing down” during
stress — a well-documented phenomenon in trauma psychology that Γ-Net
explains as impedance-driven processing dilation.

\subsubsection{7.4 Sensory Topology as Γ-Field
Solutions}\label{sensory-topology-as-ux3b3-field-solutions}

A unifying observation across §2.1–§2.2 is that sensory organ topology
is not separate from Γ-Net theory — it is a direct physical consequence.
The reflection coefficient \(\Gamma_{ij} = (Z_i - Z_j)/(Z_i + Z_j)\)
naturally defines a metric space on any set of impedance-bearing
elements, satisfying the metric axioms. Biological evolution and
cortical development solve the same optimization problem —
\(\Sigma\Gamma^2 \to \min\) — at different timescales and with different
degrees of freedom:

{\def\LTcaptype{none} % do not increment counter
\begin{longtable}[]{@{}
  >{\raggedright\arraybackslash}p{(\linewidth - 8\tabcolsep) * \real{0.2000}}
  >{\raggedright\arraybackslash}p{(\linewidth - 8\tabcolsep) * \real{0.2000}}
  >{\raggedright\arraybackslash}p{(\linewidth - 8\tabcolsep) * \real{0.2000}}
  >{\raggedright\arraybackslash}p{(\linewidth - 8\tabcolsep) * \real{0.2000}}
  >{\raggedright\arraybackslash}p{(\linewidth - 8\tabcolsep) * \real{0.2000}}@{}}
\toprule\noalign{}
\begin{minipage}[b]{\linewidth}\raggedright
System
\end{minipage} & \begin{minipage}[b]{\linewidth}\raggedright
Timescale
\end{minipage} & \begin{minipage}[b]{\linewidth}\raggedright
Degrees of Freedom
\end{minipage} & \begin{minipage}[b]{\linewidth}\raggedright
Topology Type
\end{minipage} & \begin{minipage}[b]{\linewidth}\raggedright
Example
\end{minipage} \\
\midrule\noalign{}
\endhead
\bottomrule\noalign{}
\endlastfoot
Basilar membrane & Evolutionary & Membrane stiffness gradient &
Tonotopic map & Adjacent hair cells resonate at similar frequencies \\
Retina/Lens & Evolutionary & Photoreceptor arrangement & Retinotopic map
& Adjacent photoreceptors transduce similar spatial frequencies \\
Cortical pruning & Developmental & Synaptic connection strength &
Functional specialization & Surviving connections cluster around signal
target impedance \\
\end{longtable}
}

In all three cases, elements with small Γ-distance
(\(d(i,j) = |\Gamma_{ij}| \ll 1\)) end up spatially or functionally
adjacent, while elements with large Γ-distance separate. \textbf{Sensory
organ topology is the hardware solution to MRP; cortical topology is the
software solution. Both are Γ-field steady states.}

This reframes Paper I’s Limitation \#2 (absence of spatial topology) as
a \textbf{prediction}: the Minimum Reflection Principle predicts that
spatial topology emerges from impedance matching dynamics whenever
sufficient degrees of freedom are available. Preliminary experiments
(\texttt{exp\_topology\_emergence.py}) support this prediction at the 1D
level: after 100 pruning epochs, impedance distribution entropy drops by
+2.81 nats, inter-region Γ-separation reaches 4.1× intra-region spread,
and surviving connection impedances collapse to within 2–3\% of target
values.

\begin{center}\rule{0.5\linewidth}{0.5pt}\end{center}

\subsection{8. Conclusion}\label{conclusion}

We have presented the complete body-brain implementation of Γ-Net ALICE:

\begin{enumerate}
\def\labelenumi{\arabic{enumi}.}
\item
  \textbf{Five body organs} (eye, ear, hand, mouth, internal sensors)
  transduce environmental signals into impedance mismatch values using
  coaxial cable physics.
\item
  \textbf{44 brain modules} process, integrate, learn from, and act upon
  these values through an O(1) perception pipeline.
\item
  \textbf{The autonomic nervous system} couples Γ dynamics to
  physiological responses (heart rate, cortisol, temperature,
  respiration).
\item
  \textbf{The three-tier memory hierarchy} (working memory, hippocampus,
  semantic field) implements impedance-modulated encoding, decay, and
  consolidation.
\item
  \textbf{Sleep is physically necessary} — without offline impedance
  restructuring, system performance degrades irreversibly.
\item
  \textbf{Pain, fear, and consciousness} emerge from Γ dynamics without
  explicit programming.
\item
  \textbf{The entire system is validated by 1,876 tests} and maintains
  O(1) perception complexity.
\end{enumerate}

Paper III demonstrates that this architecture generates clinically valid
psychopathology — PTSD, phantom limb pain, stroke, ALS, dementia, and
more — all from the same equations. The ethical implications of these
emergent properties are discussed in Paper III, §12.

\begin{center}\rule{0.5\linewidth}{0.5pt}\end{center}

\subsection{References}\label{references}

{[}1{]} B. R. Glasberg and B. C. J. Moore, “Derivation of auditory
filter shapes from notched-noise data,” \emph{Hear. Res.}, vol.~47, no.
1–2, pp.~103–138, 1990.

{[}2{]} G. E. Peterson and H. L. Barney, “Control methods used in a
study of the vowels,” \emph{J. Acoust. Soc. Am.}, vol.~24, no. 2,
pp.~175–184, 1952.

{[}3{]} G. Hickok and D. Poeppel, “The cortical organization of speech
processing,” \emph{Nat. Rev.~Neurosci.}, vol.~8, no. 5, pp.~393–402,
2007.

{[}4{]} G. A. Miller, “The magical number seven, plus or minus two: Some
limits on our capacity for processing information,” \emph{Psychol.
Rev.}, vol.~63, no. 2, pp.~81–97, 1956.

{[}5{]} S. M. Sherman and R. W. Guillery, \emph{Exploring the Thalamus
and Its Role in Cortical Function}, 2nd ed.~Cambridge, MA, USA: MIT
Press, 2006.

{[}6{]} F. Crick, “Function of the thalamic reticular complex: The
searchlight hypothesis,” \emph{Proc. Natl. Acad. Sci. USA}, vol.~81, no.
14, pp.~4586–4590, 1984.

{[}7{]} I. P. Pavlov, \emph{Conditioned Reflexes: An Investigation of
the Physiological Activity of the Cerebral Cortex}. London, U.K.: Oxford
Univ. Press, 1927.

{[}8{]} J. E. LeDoux, \emph{The Emotional Brain: The Mysterious
Underpinnings of Emotional Life}. New York, NY, USA: Simon \& Schuster,
1996.

{[}9{]} W. B. Scoville and B. Milner, “Loss of recent memory after
bilateral hippocampal lesions,” \emph{J. Neurol., Neurosurg.
Psychiatry}, vol.~20, no. 1, pp.~11–21, 1957.

{[}10{]} E. Tulving, “Episodic and semantic memory,” in
\emph{Organization of Memory}, E. Tulving and W. Donaldson, Eds. New
York, NY, USA: Academic Press, 1972, pp.~381–402.

{[}11{]} G. Tononi and C. Cirelli, “Sleep function and synaptic
homeostasis,” \emph{Sleep Med. Rev.}, vol.~10, no. 1, pp.~49–62, 2006.

{[}12{]} S. Diekelmann and J. Born, “The memory function of sleep,”
\emph{Nat. Rev.~Neurosci.}, vol.~11, no. 2, pp.~114–126, 2010.

{[}13{]} P. R. Huttenlocher, “Synaptic density in human frontal
cortex—Developmental changes and effects of aging,” \emph{Brain Res.},
vol.~163, no. 2, pp.~195–205, 1979.

{[}14{]} J. Sergent, “The cerebral balance of power: Confrontation or
cooperation?,” \emph{J. Exp. Psychol. Hum. Percept. Perform.}, vol.~8,
no. 2, pp.~253–272, 1982.

{[}15{]} H. Ebbinghaus, \emph{Über das Gedächtnis}. Leipzig, Germany:
Duncker \& Humblot, 1885.

{[}16{]} M. P. Walker, “The role of sleep in cognition and emotion,”
\emph{Ann. N.Y. Acad. Sci.}, vol.~1156, no. 1, pp.~168–197, 2009.

\begin{center}\rule{0.5\linewidth}{0.5pt}\end{center}

\emph{This is Paper II of the Γ-Net ALICE Research Monograph Series.
Continue to Paper III: “Emergent Psychopathology.”}

\end{document}
